% ==========================================
% BAB I PENDAHULUAN
% ==========================================
\chapter{PENDAHULUAN}
\label{chap:pendahuluan}
% --- Latar Belakang ---
\section{Latar Belakang}
Indonesia merupakan negara dengan kawasan hutan tropis terbesar ketiga di dunia
dengan keanekaragaman hayatinya yang tergolong tinggi. Namun, ancaman terhadap kebakaran
hutan kerap terjadi di Indonesia dan memberikan dampak terhadap ekologi, ekonomi, 
lingkungan, kesehatan, dan sosial \autocite{Karurung2025}. Meskipun tren karhutla telah
mengalami penurunan sesuai yang ditunjukkan pada Gambar \ref{gambar:karhutla2015-2024}, kebakaran hutan dan lahan (karhutla)
tetap menjadi salah satu sumber utama emisi karbon. Selain itu, kejadian deforestasi hutan
sering kali berhubungan erat dengan bencana karhutla \autocite{Vetrita2025}.

\begin{figure}[h] % pilihan opsi yang disarankan: t = top, b = bottom, h = here
	\centering
 	\captionsetup{justification=centering}
    	\includegraphics[width=1\textwidth]{image/data_karhutla_2015-2024.png}
	\caption{Karhutla di Indonesia pada tahun 2015--2024 \cite{sipongi_klhk}}
	\label{gambar:karhutla2015-2024}
\end{figure}

Kebakaran hebat terjadi terutama saat musim kemarau dan tahun-tahun kering. Berdasarkan data 
dari Kementerian Lingkungan Hidup dan Kehutanan (KLHK), Indonesia mengalami kebakaran hebat
yang membakar 2,6 juta hektare hutan dan lahan yang dipicu oleh kekeringan akibat El Niño pada tahun 2015. 
Peristiwa tersebut menghasilkan emisi karbon dalam jumlah besar
disertai kabut berkepanjangan di seluruh Indonesia dan negara tetangga dengan total kerugian
ekonomi yang berkisar antara 1,62 hingga 2,7 miliar USD (United States Dollars) \autocite{Prasetyo2022}. Selain kerugian ekonomi, 
kebakaran hutan juga memberikan dampak yang besar bagi kesehatan dan lingkungan.

Sebagaimana yang dijelaskan pada \textcite{Hein2022}, kebakaran hutan pada tahun 2015 menghasilkan 
partikel PM\textsubscript{2.5} (partikel halus berdiameter 2,5 mikrometer yang biasanya 
merupakan asap berbahaya) selama 24 jam dengan jumlah di atas 2.000~µg/m\textsuperscript{3}.
Partikel ini memberikan dampak buruk pada kesehatan seperti memperburuk asma, memicu gejala pernapasan, 
meningkatkan angka rawat inap, hingga berkontribusi terhadap kanker paru-paru. Pada jangka panjang, 
kebakaran hutan dapat merusak lahan serta menjadi bencana yang menimbulkan hilangnya biodiversitas dan 
menyebabkan pemanasan global \autocite{Thoha2021}.

Menurut studi dari \textcite{Karurung2025}, penyebab kebakaran hutan dapat beragam mulai dari fase ENSO (\textit{El Niño-Southern Oscillation}), 
curah hujan di sekitar Juli hingga September, faktor meteorologi dan topografi, hingga aktivitas
manusia. Meninjau dari beberapa penyebab tersebut, pemerintah dan beberapa penelitian telah melakukan usaha untuk 
melakukan penanganan dan pencegahan terhadap karhutla. Penelitian yang dilakukan oleh \textcite{Qamariyanti2023} menunjukkan
bahwa di Indonesia sudah terdapat beberapa upaya pencegahan yang dilakukan oleh pemerintah pada tahun 2020 seperti
membuat peraturan dan pasal terhadap tindakan pembukaan lahan dengan kebakaran hutan. Selain itu, Badan Meteorologi, Klimatologi, dan Geofisika (BMKG)
juga telah membuat pendekatan identifikasi karhutla melalui perhitungan Fire Weather Index yang memanfaatkan faktor 
meteorologis dalam melakukan identifikasi. Walaupun begitu, Fire Weather Index yang dipandang sebagai perhitungan indeks kebakaran
yang telah teruji secara internasional belum mengintegrasikan faktor-faktor lainnya yang menurut beberapa studi menjadi
penyebab terjadinya bencana karhutla di Indonesia.

Di sisi lain, pemantauan karhutla di Indonesia sudah mulai bergantung pada data satelit \textit{hotspot} harian dari produk 
Visible Infrared Imaging Radiometer Suite (VIIRS) 375 m. Produk ini kemudian diintegrasikan dengan sistem informasi kebakaran nasional.
Namun, studi oleh \textcite{Indradjad2024} menemukan bahwa data VIIRS hanya menemukan sekitar 52--53\% kejadian lapangan yang memiliki \textit{hotspot}, 
sedangkan untuk kebakaran yang lebih besar dari 14 hektare tingkat deteksinya meningkat hingga sekitar 83\%. Hal ini menunjukkan masih 
adanya beberapa kebakaran kecil dan menengah yang tidak terdeteksi oleh satelit. 

Dalam beberapa tahun terakhir, teknologi \textit{remote sensing} dan \textit{geospatial} telah muncul sebagai
alat untuk mendeteksi dan memetakan kebakaran menggunakan faktor alam. Bahkan, pengembangan teknologi telah mencapai
pemanfaatan integrasi \textit{Machine Learning} (ML) dalam melakukan pemantauan dengan lebih efektif. Akan tetapi, 
integrasi data satelit dengan ML atau \textit{Deep Learning} (DL) untuk 
prediksi dan pemodelan kebakaran masih merupakan bidang yang relatif jarang dieksplorasi di wilayah Asia Tenggara
dan memiliki potensi besar untuk dikembangkan \autocite{Eaturu2025}.

Selain itu, pemanfaatan model ML dalam tindakan prediktif kebakaran hutan sering kali hanya 
berfokus kepada hasil atau akurasi dari model. Padahal dalam memberikan informasi, alasan dalam penentuan 
hasil prediktif sangat diperlukan untuk memberikan wawasan yang signifikan. Kurangnya transparansi saat 
menggunakan model ML merupakan hambatan bagi para pengelola kebakaran hutan karena 
algoritma ini dianggap sebagai "\textit{black box}" sehingga sulit memahami hasil yang diberikan oleh model \autocite{Abdollahi2023}.
Maka dari itu, pendekatan \textit{Explainable Artificial Intelligence} (XAI) perlu diterapkan dalam konteks
kebakaran hutan untuk menginterpretasikan kontribusi dari faktor-faktor terhadap hasil model. 

% --- Rumusan Masalah ---
\section{Rumusan Masalah}
Berdasarkan latar belakang yang telah dijelaskan dapat diketahui bahwa saat ini langkah prediktif terhadap kebakaran hutan masih dalam tahap penelitian dan kurang dieksplorasi 
khususnya di wilayah Asia Tenggara. Selain itu, sistem yang sudah ada saat ini terkadang menghambat pengambilan keputusan pemangku kepentingan, khususnya pada langkah mitigasi. Model prediktif 
dapat dimanfaatkan untuk memberikan wawasan dan acuan dasar dalam menyusun langkah mitigasi seperti penentuan wilayah patroli. Meskipun BMKG saat ini menyediakan indeks bahaya kebakaran berbasis 
Fire Weather Index yang valid secara meteorologis untuk pemantauan harian, indeks yang tersedia masih terbatas pada variabel meteorologis dan belum mengintegrasikan komponen biofisik maupun antropogenik.
Walaupun sistem tersebut sudah menerapkan pendekatan \textit{forecasting}, faktor yang digunakan masih berfokus ke faktor meteorologis.

Oleh karena itu, langkah prediktif diperlukan untuk memberikan informasi terkait indikator karhutla dengan menggunakan faktor yang lebih luas 
dan relevan menurut banyak penelitian sekaligus memberikan acuan informasi yang lebih cepat sebagai bahan perancangan mitigasi. Namun, 
pendekatan prediktif seperti memanfaatkan ML sering kali menimbulkan keraguan disebabkan hasil prediksi yang diberikan bersifat \textit{black box} sehingga hasil
yang diberikan masih diragukan oleh pemangku kepentingan. Berdasarkan penjelasan sebelumnya, didefinisikan beberapa rumusan masalah sebagai berikut.
\begin{enumerate}
	\item	Bagaimana mengembangkan model prediksi risiko kebakaran hutan dan lahan berbasis ML dengan memanfaatkan faktor-faktor penyebab karhutla?
	\item	Bagaimana cara menerapkan pendekatan XAI untuk menjelaskan peran faktor-faktor utama yang mempengaruhi hasil prediksi risiko terjadinya karhutla?
	\item	Bagaimana bentuk penerjemahan hasil prediksi sehingga dapat memberikan wawasan yang relevan untuk mendukung keputusan mitigasi?
\end{enumerate}

% --- Tujuan ---
\section{Tujuan}
Berdasarkan rumusan masalah yang telah dituliskan pada bagian sebelumnya, berikut adalah tujuan utama dari pengerjaan tugas akhir ini.

\begin{enumerate}
	\item	Membangun model prediksi risiko kebakaran hutan dan lahan berbasis ML dengan memanfaatkan data hotspot satelit, informasi meteorologi, dan faktor lingkungan pada suatu wilayah studi di Indonesia.
	\item	Menerapkan pendekatan XAI untuk menginterpretasikan faktor-faktor utama yang memengaruhi hasil prediksi risiko karhutla pada model yang dikembangkan.
	\item	Merancang prototipe visualisasi spasial yang menyajikan hasil prediksi risiko karhutla secara informatif dan mudah dipahami sebagai bahan pertimbangan dalam upaya pencegahan dan mitigasi.
\end{enumerate}

Keberhasilan tugas akhir ini dievaluasi melalui pencapaian akurasi model yang melampaui 0,80 dalam memprediksi titik api dan nilai F1-score mencapai 0,75 untuk kelas risiko tinggi serta kemampuan sistem dalam memberikan alasan di balik hasil prediksi model dan 
penyajian informasi secara informatif dan jelas yang dibungkus dalam bentuk prototipe visual. Selain itu, nilai \textit{fidelity} yang ditargetkan untuk XAI adalah sekitar 0,80 sehingga 
penjelasan yang dihasilkan diharapkan cukup konsisten dengan keputusan model. Selain itu, tugas akhir ini tidak ditujukan sebagai langkah untuk menggantikan sistem Fire Weather Index yang telah
digunakan saat ini, tugas akhir ini akan tetap memanfaatkan indeks tersebut sebagai salah satu masukan dalam model prediksi risiko terjadinya karhutla.
% --- Batasan Masalah ---
\section{Batasan Masalah}
Adapun untuk batasan masalah dalam pengerjaan tugas akhir ini adalah sebagai berikut.

\begin{enumerate}
	\item	Untuk mengurangi masalah keterbatasan data dan beban infrastruktur pengembangan, tugas akhir difokuskan pada wilayah Kalimantan dan menggunakan model ML.
	\item	Jenis data yang digunakan dibatasi pada data satelit hotspot, data meteorologi, dan data biofisik yang tersedia secara terbuka.
	\item	Prototipe penyajian hasil prediksi bersifat demonstratif dan berbasis data historis yang telah diproses secara \textit{offline}. 
	\item	Cakupan waktu yang diprediksi adalah skala bulanan untuk mengurangi kompleksitas data dan mendukung perencanaan strategis.
\end{enumerate}
% --- Metodologi Pengerjaan TA ---
\section{Metodologi}
Metodologi yang digunakan dalam tugas akhir ini mengacu pada kerangka kerja CRISP-DM (\textit{Cross-Industry Standard Process for Data Mining}).
CRISP-DM dipilih karena menyediakan tahapan yang sistematis untuk mengembangkan solusi berbasis data yang dimulai dari pemahaman masalah hingga pembangunan model.
Metodologi ini diadaptasi agar sesuai dengan konteks pemodelan risiko karhutla dan pengembangan prototipe visual yang disertai penerapan
XAI.

\begin{figure}[h] % pilihan opsi yang disarankan: t = top, b = bottom, h = here
	\centering
  \captionsetup{justification=centering}
    	\includegraphics[width=1\textwidth]{image/crisp-dm.png}
	\caption{Metodologi CRISP-DM}
	\label{gambar:metodologi_crisp-dm}
\end{figure}

Menurut Gambar \ref{gambar:metodologi_crisp-dm}, tahapan metodologi yang akan dilaksanakan adalah sebagai berikut.

\begin{enumerate}
	\item	\textit{Business Understanding} \\
	Pada tahap ini dilakukan pemahaman terhadap konteks kebakaran hutan dan lahan di wilayah studi beserta dampak
	yang ditimbulkan. Kegiatan yang dilakukan meliputi studi literatur mengenai karhutla di Indonesia, penyebab dan
	dampak dari karhutla, serta identifikasi kebutuhan akan informasi risiko sebagai langkah pemenuhan
	tindakan mitigasi. Hasil dari tahap ini adalah pemahaman konteks, perumusan masalah dan tujuan, pemodelan,
	ruang lingkup wilayah studi, serta kriteria keberhasilan pengembangan model dan prototipe. Selain itu, langkah pemahaman
	diawali dengan melakukan riset dan kajian untuk memahami permasalahan dan pendekatan solusi yang dilakukan dengan memanfaatkan 
	artikel-artikel terbitan tahun 2021 ke atas agar memiliki relevansi yang signifikan. Artikel yang digunakan diusahakan memiliki reputasi yang baik dan berasal
	dari jurnal yang terindeks Scopus atau SINTA. Namun, beberapa artikel yang terbit di bawah tahun 2021 terkadang digunakan
	sebagai dasar atau dukungan studi.
	\item	\textit{Data Understanding} \\
	Tahap ini berfokus pada pengenalan dan eksplorasi data yang akan digunakan. Data yang dimaksud mencakup 
	data \textit{hotspot} atau kebakaran dari satelit, data meteorologi, serta data lingkungan/biofisik wilayah studi. 
	Kegiatan meliputi identifikasi sumber data, pengumpulan data historis, analisis struktur dan format data, 
	serta eksplorasi awal pola spasial dan temporal kejadian karhutla. Selain itu, dilakukan pula penilaian awal 
	terhadap kualitas data, seperti keberadaan nilai hilang, inkonsistensi, atau ketidakseimbangan kelas antara 
	kejadian kebakaran dan non-kebakaran.
	\item	\textit{Data Preparation} \\
	Pada tahap ini dilakukan pengolahan data agar siap digunakan dalam pemodelan. Kegiatan yang dilakukan antara 
	lain pemilihan periode waktu yang akan dianalisis, pembatasan data pada wilayah studi, pembentukan representasi spasial 
	(pembentukan \textit{grid}), penggabungan data hotspot dengan data meteorologi dan data lingkungan, serta 
	pembersihan data (penanganan nilai hilang, duplikasi, dan nilai yang tidak wajar). Pada tahap ini juga dilakukan
	analisis terhadap fitur yang relevan dengan risiko karhutla serta pembagian data menjadi pelatihan dan pengujian.
	\item	\textit{Modelling} \\
	Tahap pemodelan berfokus pada pembangunan model \textit{Supervised Machine Learning} untuk memprediksi risiko kebakaran hutan dan
	lahan. Kegiatan meliputi pemilihan model yang sesuai, skema pelatihan dan validasi, serta penyesuaian parameter model.
	Pada tahap ini juga dilakukan penerapan teknik XAI sebagai langkah untuk menjelaskan kontribusi masing-masing fitur terhadap
	hasil prediksi.
	\item	\textit{Evaluation} \\
	Evaluasi model dilakukan dengan meninjau nilai kuantitatif yang dihasilkan dengan menggunakan beberapa metrik utama yang umum
	digunakan pada evaluasi model, termasuk \textit{accuracy}, \textit{precision}, \textit{recall}, \textit{F1-Score}, serta\textit{Area Under ROC Curve} (AUC-ROC). 
	Selain itu, evaluasi kualitatif juga dilakukan berdasarkan hasil interpretasi dari XAI untuk menilai apakah faktor-faktor
	yang muncul selaras dengan domain terkait karhutla di wilayah studi.
	\item	\textit{Deployment} \\
	Pada tahap terakhir, hasil pemodelan dan interpretasi XAI disusun dalam bentuk prototipe penyajian hasil prediksi berupa 
	visualisasi spasial yang menampilkan peta risiko karhutla pada wilayah studi. Prototipe ini dirancang agar informasi risiko 
	dan faktor-faktor penjelas dapat disajikan secara sederhana dan objektif sehingga mudah dipahami oleh pengguna.

\end{enumerate}
