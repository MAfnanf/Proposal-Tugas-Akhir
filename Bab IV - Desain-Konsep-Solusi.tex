% ==========================================
% BAB IV DESAIN KONSEP SOLUSI
% ==========================================
\chapter{DESAIN KONSEP SOLUSI}
\label{chap:desain-konsep-solusi}
Bab ini menjelaskan tentang desain yang diajukan berdasarkan hasil penentuan solusi alternatif pada Bab III. Penjelasan yang dilakukan
berupa ilustrasi diagram sehingga terdapat konsep sistem yang menggambarkan bentuk sistem solusi. 

\begin{figure}[h] % pilihan opsi yang disarankan: t = top, b = bottom, h = here
	\centering
  \captionsetup{justification=centering}
    	\includegraphics[width=0.8\textwidth]{image/model-konseptual-to-be.png}
	\caption{Model konseptual \textit{to-be}}
	\label{gambar:model-konseptual-tobe}
\end{figure}

\begin{figure}[t] % pilihan opsi yang disarankan: t = top, b = bottom, h = here
	\centering
  \captionsetup{justification=centering}
    	\includegraphics[width=1\textwidth]{image/sistem-prediktif.png}
	\caption{Sistem prediktif dengan ML, XAI, dan penyajian informasi visual}
	\label{gambar:sistem-prediktif}
\end{figure}

Berdasarkan analisis masalah dan pemilihan alternatif pada Bab III, solusi yang dipilih adalah pengembangan sistem prediksi potensi
karhutla berbasis ML dan XAI yang menyajikan informasi risiko dalam bentuk
peta dan ringkasan faktor utama.
Sistem ini ditujukan untuk melengkapi sistem pemantauan yang sudah ada sehingga 
pemangku kepentingan dapat memperoleh informasi risiko yang lebih proaktif untuk
mendukung penentuan prioritas mitigasi.

Gambar \ref{gambar:model-konseptual-tobe} menunjukkan model konseptual setelah penambahan solusi usulan (\textit{to-be}). Meninjau dari 
kondisi saat ini (\textit{as-is}) yang ditunjukkan pada Gambar \ref{gambar:model-konseptual-saat-ini}, langkah yang dilakukan masih bersifat reaktif setelah mendapatkan 
acuan informasi titik api berdasarkan data dari satelit untuk merencanakan tindakan pengelolaan seperti 
patroli. Namun, kondisi \textit{to-be} yang ditandai dengan kotak berwarna oranye memberikan pendekatan yang berbeda
dari kondisi \textit{as-is}, yaitu menggunakan model prediktif dalam memberikan informasi yang lebih cepat.
Modul prediksi menerima data yang sudah dikumpulkan, menghasilkan level risiko karhutla per wilayah analisis, dan 
mengirimkannya ke sistem pemantauan sehingga indikator potensi karhutla kini mempertimbangkan informasi risiko prediktif
di samping kondisi yang sedang berlangsung.

Gambar \ref{gambar:sistem-prediktif} menunjukkan konsep desain sistem prediksi potensi
karhutla yang diusulkan. Secara ringkas, tahapan utama sistem dijelaskan sebagai berikut.
\begin{enumerate}
    \item Pengumpulan data historis bulanan \\
    Sistem mengumpulkan data historis titik \textit{hotspot}, faktor antropogenik, faktor biofisik, dan
    faktor meteorologis untuk setiap \textit{grid} pada wilayah Provinsi Kalimantan. Seluruh data dikumpulkan 
    dalam satuan waktu bulanan sehingga setiap rekaman mempresentasikan kombinasi \textit{grid}-bulan. Skala
    bulanan dipilih karena relevan dengan perencanaan operasional mitigasi sekaligus lebih efisien
    dari sisi volume dan pengolahan data
    \item Pemrosesan dan integrasi data \\
    Data mentah dari berbagai sumber akan melalui tahap pemrosesan seperti pembersihan nilai dan penyesuaian
    format. Selain itu, langkah filter dan integrasi data juga dilakukan untuk mengetahui penyebab suatu \textit{grid}
    memiliki titik api atau tidak. Hasil tapah ini adalah \textit{dataset} bulanan per \textit{grid} yang siap digunakan
    untuk pelatihan dan evaluasi model
    \item Pemodelan prediktif berbasis ML \\
    \textit{Dataset} terproses akan dijadikan masukan bagi model ML agar dapat menjadi model prediktif karhutla.
    Model ML yang digunakan adalah RF dan XGBoost agar memiliki \textit{benchmark} dan dapat menentukan model yang terbaik.
    Model berbasis \textit{ensemble tree} sangat relevan untuk memetakan hubungan kombinasi dari faktor meteorologis,
    biofisik, dan antropogenik. Output utama dari model ini adalah skor atau kelas (level) risiko karhutla bulanan untuk setiap 
    \textit{grid} wilayah. Resolusi waktu prediksi bulanan ini dipilih untuk mendukung perencanaan strategis dan manajemen logistik yang 
    secara administratif tidak bisa dilakukan dalam siklus harian. Jadi, prediksi ini melengkapi sistem harian BMKG yang bersifat taktis
    dan operasional.
    \item Analisis faktor menggunakan XAI \\
    Untuk memberikan penjelasan dari hasil model, sistem menerapkan pendekatan XAI dengan memanfaatkan SHAP sebagai 
    interpretabilitas lokal dan PFI untuk interpretabilitas global. Metode ini menghasilkan informasi kontribusi 
    masing-masing faktor terhadap prediksi risiko sehingga memberikan penjelasan dari hasil model.
    \item Penyajian informasi kepada pemangku kepentingan \\
    Hasil dari model memberikan kerentanan risiko bulanan dengan level rendah, sedang, dan tinggi yang disertai dengan hasil
    analisis XAI dan disajikan dalam bentuk peta grafik risiko. Penyajian ini bertujuan agar pemangku kepentingan seperti 
    instansi pengelola karhutla dapat memahami hasil prediksi dan menyusun rencana mitigasi dengan lebih cepat karena 
    dapat memanfaatkan hasil prediksi sebagai acuan dasar. Walaupun prediksi bukanlah hasil yang pasti, hasil yang diberikan
    model tidak sekedar sebuah hasil. Namun, hasil yang diberikan disertai penjelasan sehingga analisis lanjutan mungkin 
    tetap perlu dilakukan sebelum menyusun rencana mitigasi.
\end{enumerate}

Berdasarkan penjelasan kondisi sistem \textit{as-is} dan \textit{to-be} sebelumnya, analisis penjelasan perbandingan dilakukan sesuai
pada Tabel \ref{tbl:perbandingan-asis-tobe}. Tabel \ref{tbl:perbandingan-asis-tobe} merangkum perbedaan utama antara sistem \textit{as-is} dan sistem \textit{to-be}.
Secara singkat, sistem usulan tidak menggantikan mekanisme pemantauan yang sudah ada, tetapi menambahkan modul prediktif
bulanan dan penjelasan faktor yang diharapkan dapat menjawab kebutuhan informasi yang diidentifikasi pada Bab III.

\begin{table}[h]
    \centering
    \begin{tabular}{| p{2.5cm} | p{5cm} | p{5cm} |}
    \hline
    Aspek & \textit{As-Is} (saat ini) & \textit{To-Be} (solusi usulan) \\
    \hline
    Jenis informasi utama &
    \textit{Hotspot} dan indeks bahaya kebakaran yang menggambarkan kondisi saat ini. &
    Level risiko karhutla bulanan per \textit{grid} yang dibangun dari data historis multi-faktor. \\
    \hline
    Fungsi utama &
    Mendukung monitoring dan respon reaktif ketika bahaya sudah tinggi. &
    Menambah dukungan untuk perencanaan mitigasi proaktif melalui peta risiko dan informasi level potensi kebakaran. \\
    \hline
    Pemahaman faktor penyebab &
    Analisis faktor meteorologis, biofisik, dan antropogenik dilakukan terpisah dan bersifat kualitatif. Fokus saat ini hanya menggunakan FWI sebagai identifikasi faktor meteorologis. &
    \textit{Explainable AI} memberikan kontribusi relatif tiap faktor terhadap risiko, baik pada tingkat lokal maupun global. \\
    \hline
    Bentuk penyajian &
    Portal pemantauan dan laporan operasional. &
    Visualisasi dalam bentuk peta risiko bulanan. \\
    \hline
    \end{tabular}
    \caption{Perbandingan Ringkas Sistem \textit{As-Is} dan \textit{To-Be}}
    \label{tbl:perbandingan-asis-tobe}
\end{table}

