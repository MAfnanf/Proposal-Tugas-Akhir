% ============================================================================================
% BAB III ANALISIS MASALAH
% Pembagian subbab tidak rigid dan dapat bervariasi. Bab ini minimal berisi analisis kebutuhan
% fungsional dan nonfungsional, analisis berbagai alternatif solusi yang dapat ditawarkan, dan
% metode pemilihan solusi yang diusulkan.
% ============================================================================================
\chapter{ANALISIS MASALAH}
\label{chap:analisis-masalah}
\section{Analisis Kondisi}
Bab ini membahas kondisi aktual dari karhutla di Indonesia, khususnya di wilayah 
Kalimantan serta bagaimana kondisi mempengaruhi kebutuhan akan sistem prediksi 
risiko karhutla berbasis data. Analisis yang dilakukan mengikuti alur metodologi dari 
CRISP-DM, yaitu dimulai dari pemahaman terhadap konteks masalah,
proses bisnis, dan pemangku kepentingan yang terlibat dalam pengendalian bencana karhutla
(\textit{business understanding}). Selanjutnya, pemahaman terhadap data yang digunakan
untuk memahami karakteristik seputar karhutla dilakukan untuk memahami bagaimana 
solusi-solusi yang telah diterapkan saat ini dalam menghadapi langkah mitigasi karhutla
(\textit{data understanding}).

Melalui analisis kondisi ini, diharapkan kondisi kesenjangan (\textit{gap}) antara kebutuhan 
lapangan dengan kemampuan sistem dapat terlihat dengan jelas. Kesejangan inilah yang kemudian
akan dimanfaatkan untuk menjadi dasar dalam perumusan kebutuhan sistem, perancangan alternatif
solusi, serta pemilihan pendekatan yang paling relevan untuk dikembangkan pada tugas akhir
ini.

\subsection{\textit{Business Understanding}}
\subsubsection{Kondisi Saat Ini}
Walaupun Indonesia merupakan salah satu negara tropis terbesar di dunia, ancaman terhadap karhutla
masih terjadi secara berulang dan menimbulkan dampak yang luas terhadap ekologi, ekonomi, kesehatan, dan
sosial. Karhutla di Indonesia tidak hanya dipengaruhi oleh satu faktor, melainkan merupakan gabungan atau
hasil interkasi antara variabilitas iklim, kondisi biofisik, dan aktivitas manusia sebagaimana telah dibahas
pada Bab II. Hal ini membuat pola terjadinya karhutla menjadi kompleks, berbeda antar wilayah, dan berubah dari
waktu ke waktu.

Saat ini, beberapa lembaga di Indonesia telah membuat sistem untuk mengendalikan dan mengelola bencana karhutla
sebagai langkah pemantauan dan peringatan dini. Kementerian Lingkungan Hidup Dan Kehutanan (KLHK) mengembangkan 
sebuah sistem melalui SiPongi+ dengan menyajikan informasi \textit{hotspot} saat ini dan luas kebakaran melalui sebuah situs web 
interaktif yang dapat diakses secara terbuka. Sementara itu, BMKG menyediakan indeks bahaya kebakaran melalui perhitungan
matematis, seperti \textit{Fire Weather Index} (FWI), \textit{Initial Spread Index} (ISI), dan lainnya.
Sistem-sistem ini sangat penting untuk memantau titik api dan kondisi bahaya kebakaran secara harian untuk mendukung
respon reaktif yang cepat di lapangan. Namun dari sudut pandang pengelolaan risiko kebakaran, masih terdapat sejumlah
tantangan yang dihadapi pemangku kepentingan, khususnya ketika mereka ingin menyusun langkah mitigasi yang lebih 
proaktif dan terarah karena kurangnya informasi dengan jangka waktu yang lebih panjang.

Secara ringkas, alur pengendalian karhutla yang berjalan di Indonesia saat ini digambarkan pada model konseptual pada Gambar \ref{gambar:model-konseptual-saat-ini}.
Pembuatan model konseptual tersebut dilakukan dengan memanfaatkan informasi yang didapatkan pada situs resmi KLHK dan berdasarkan 
studi dari \textcite{Prayoga2021} sehingga memiliki validasi terhadap kondisi saat ini.
\begin{figure}[h] % pilihan opsi yang disarankan: t = top, b = bottom, h = here
	\centering
  \captionsetup{justification=centering}
    	\includegraphics[width=1\textwidth]{image/model-konseptual-saat-ini.png}
	\caption{Konsep sistem penanganan karhutla di Indonesia saat ini}
	\label{gambar:model-konseptual-saat-ini}
\end{figure}

Proses tersebut diawali oleh kondisi iklim dan aktivitas pembukaan lahan yang kemudian akan meningkatkan potensi munculnya titik api. 
Langkah selanjutnya adalah pengumpulan data titik api, cuaca, dan kondisi lahan gambut yang didapatkan dari penyajian informasi lewat sistem
pemantauan nasional seperti SiPongi+. Informasi tersebut digunakan untuk menilai apakah indikator menunjukkan potensi karhutla sehingga terdapat
langkah tindak lanjut yang akan dilakukan. Apabila kebakaran belum terkendali, terdapat langkah penguatan operasi. Namun bila sudah terkendali, tahap evaluasi
akan dilakukan. Langkah ini memiliki satu kelemahan efektivitas, yaitu keputusan penanggulangan baru direncanakan apabila titik api telah terdeteksi sehingga
bisa menghambat efektivitas penanggulangan. \textcite{Prayoga2021} juga menilai bahwa alur ini cenderung berorientasi pada respon sehingga 
diperlukan penguatan tahap kesiapsiagaan melalui pemanfaatan data yang lebih ilmiah dan prediktif.

Beberapa tantangan utama yang muncul antara lain:
\begin{enumerate}
    \item Lingkungan risiko yang kompleks dan dinamis. \\
    Risiko karhutla ditentukan oleh kombinasi faktor meteorologik (cuaca, suhu, angin, ENSO), faktor biofisik (jenis tutupan
    lahan, lahan gambut, topografi), dan faktor antropogenik (aktivitas manusia). Kombinasi ini membuat langkah rekognisi yang telah dilakukan
    saat ini menggunakan pendekatan WFI menjadi kurang relevan disebabkan hanya bergantung pada satu faktor, yaitu faktor meteorologik.
    \item Pola kejadian yang tidak merata secara ruang dan waktu. \\
    Data historis menunjukkan bahwa kejadian karhutla cenderung terjadi pada wilayah dan periode tertentu, misalnya pada tipe
    lahan gambut dan saat musim kemarau yang diperkuat dengan fenomena ENSO. Beberapa kawansan mengalami kebakaran berulang, Sementara
    area lain relatif jarang terbakar. Tanpa alat bantu analisis yang memanfaatkan pola historis ini secara sistematis, pengelola
    akan kesulitan mengidentifikasi wilayah mana yang rentan sehingga perlu menjadi prioritas mitigasi.
    \item Fokus informasi pada kejadian dan bahaya jangka pendek. \\
    Informasi yang disediakan oleh pemantauan yang ada umumnya berfokus pada langkah reaktif dengan memanfaatkan indeks bahaya kebakaran
    berbasis cuaca. Informasi tersebut sangat berguna untuk operasi tanggap darurat, namun belum sepenuhnya menjawab kebutuhan perencanaan
    mitigasi. Patroli yang dilakukan berdasarkan data saat ini atau historis yang mungkin saja menyebabkan keterlambatan tindakan. Hal ini
    disebabkan belum adanya acuan prediksi yang bisa dimanfaatkan untuk menentukan wilayah yang menjadi prioritas sehingga bisa dilakukan 
    operasi tanggap darurat dengan lebih cepat.    
    \item Terbatasnya pemahaman terstruktur tentang faktor penyebab. \\
    Bagi pemangku kepentingan, pertanyaan yang diajukan sebelum mengambil keputusan bukan hanya tentang "di mana risiko tertinggi saat ini",
    tetapi juga "mengapa wilayah tersebut memiliki risiko yang tinggi". Saat ini, hubungan antara faktor-faktor penyebab karhutla sering Kalimantan
    dianalisis secara terpisah. Hal tersebut akan menghambat pemangku kepentingan dalam melakukan penyusunan keputusan strategi mitigasi karhutla.
    \item Langkah penguatan operasi yang belum optimal. \\
    Untuk meningkatkan langkah operasional dalam mengendalikan bencana karhutla, pemenuhan suber daya sering kali menjadi tantangan dalam hal ini. 
    Dalam meningkatkan operasi, mobilisasi sumber daya dan rencana pengendalian membutuhkan waktu perencanaan administrasi dan eksekusi fisik yang 
    memakan waktu mingguan. Maka dari itu, diperlukan sebuah informasi dari perencanaan dari jauh-jauh hari untuk membuat keputusan strategis yang tidak 
    mendadak.
\end{enumerate}
Rangkaian tantangan di atas menunjukkan bahwa meskipun Indonesia saat ini telah memiliki beberapa sistem pemantauan dan peringatan dini, masih terdapat
kesenjangan dalam hal ketersediaan informasi risiko untuk membantu melakukan langkah mitigasi dengan mengetahui wilayah mana yang paling rentan
diserta faktor-faktor apa yang paling berkontribusi. Analisis inilah yang akan menjadi fokus pada tahap selanjutnya dalam tugas akhir untuk menentukan
langkah yang tepat untuk membantu mengatasi permasalahan ini.

\subsection{Kebutuhan Fungsional}
Berdasarkan pemahaman terhadap masalah, langkah selanjutnya adalah merumuskan kebutuhan sistem secara konseptual untuk mengetahui
bagaimana sistem yang relevan dengan permasalahan. Kebutuhan fungsional menggambarkan fungsi inti apa saja yang harus disediakan 
oleh suatu sistem informasi pendukung risiko karhutla. Kebutuhan fungsional berguna untuk menyusun dan memahami bagaimana sistem 
harus mampu untuk memberikan fitur yang dapat membantu mengatasi permasalahan. Kebutuhan fungsional yang diidentifikasikan dirangkum pada Tabel III.1.

\begin{table}[t]
  \begin{tabular}{| p{1.2cm} | p{4cm} | p{7cm} |}
    \hline
    ID & Kebutuhan Fungsional & Penjelasan singkat (dikaitkan dengan masalah) \\
    \hline
    F1 & Integrasi data karhutla dan faktor lingkungan &
    Sistem dapat mengintegrasikan data historikal karhutla dengan data faktor lingkungan yang relevan agar analisis hasil memiliki dasar yang jelas. \\
    \hline
    F2 & Analisis pola kebakaran &
    Sistem mampu mengidentifikasi pola kebakaran secara spasial dan temporal berdasarkan rangkaian data historis. \\
    \hline
    F3 & Analisis keterkaitan faktor risiko dengan kejadian &
    Sistem perlu menyediakan hasil analisis pola dan penyajian informasi analisis untuk memeberikan hasil transparansi pola. \\
    \hline
    F4 & Penyajian informasi visual &
    Sistem mampu menyajikan informasi pola dan faktor penyebab untuk memberikan wawasan kepada pemangku kepentingan. \\
    \hline
  \end{tabular}
  \caption{Kebutuhan Fungsional Sistem (versi ringkas dan problem-driven)}
  \label{tbl:kebutuhan-fungsional}
\end{table}

\subsection{Kebutuhan Non-funsional}
Kebutuhan non-fungsional tidak berbicara tentang "apa yang dilakukan" sistem, melainkan bagaimana karakteristik sistem tersebut agar relevan dengan konteks pengendalian karhutla di Indonesia. 
Dengan menganalisis kebutuhan non-fungsional dari sistem, pengetahuan terhadap kelayakan teknis dapat dipahami untuk pengembangan yang relevan. Berikut hasil analisis kebutuhan 
non-fungsional pada permasalahan terkait.

Selain itu, pada tugas akhir ini kebutuhan non-fungsional difokuskan pada tiga aspek utama, yaitu akurasi model, ketersediaan dan waktu respon sistem, serta kemudahan penggunaan. 
Akurasi model diperlukan agar informasi risiko yang dihasilkan tidak menyesatkan pengambil keputusan, ketersediaan dan waktu respon menentukan sejauh mana sistem dapat mendukung analisis 
secara interaktif, sedangkan kemudahan penggunaan memastikan bahwa sistem dapat dimanfaatkan oleh pemangku kepentingan yang tidak selalu memiliki latar belakang teknis. Rangkuman kebutuhan 
non-fungsional tersebut disajikan pada Tabel~\ref{tbl:kebutuhan-nonfungsional}.

\begin{table}[H]
  \centering
  \begin{tabular}{| p{1.2cm} | p{3cm} | p{8cm} |}
    \hline
    ID & Kebutuhan Non-fungsional & Indikator / Target \\
    \hline
    NF1 & Akurasi Model &
    Model analisis/prediksi risiko karhutla yang digunakan dalam sistem
    harus mencapai tingkat kinerja yang dapat diterima, dengan nilai
    AUC (Area Under Curve) pada data uji minimal.
    $\geq 0{,}80$. \\ \hline

    NF2 & Keandalan &
   Sistem mampu memberikan hasil deteksi atau analisis risiko karhutla
    yang konsisten dan dapat dipercaya, disertai bentuk keluaran yang jelas
    (misalnya kelas risiko atau skor risiko) sehingga tidak menimbulkan
    ambiguitas  \\ \hline

    NF3 & Kemudahan penggunaan &
    Sistem harus disajikan dalam bentuk yang mudah dipahami
    dan cukup responsif, sehingga pengguna non-teknis dapat
    mengoperasikan fitur utama. \\ \hline

  \end{tabular}
  \caption{Kebutuhan non-fungsional sistem}
  \label{tbl:kebutuhan-nonfungsional}
\end{table}

\section{Analisis Pemilihan Solusi}
Berdasarkan hasil identifikasi kebutuhan fungsional dan non-fungsional, langkah penentuan beberapa alternatif solusi dilakukan untuk
mencari solusi yang secara konseptual dapat membantu memenuhi kebutuhan tersebut. Alternatif-alternatif ini disusun dengan merangkup pendekatan
yang digunakan pada penelitian terdahulu dan membandingkannya dengan rancangan sistem yang diusulkan pada tugas akhir.
Agar pemilihan tidak hanya berdasarkan preferensi subjektif, proses evaluasi alternatif dilakukan dengan menggunakan pendekatan \textit{Multi-
Criteria Decision Analysis} (MCDA) dengan metode pembobotan sederhana yang diadaptasi dari studi \textcite{OShea2026}.

\subsection{Alternatif Solusi}
Untuk langkah prediktif karhutla, terdapat beberapa pendekatan yang digunakan, terutama dalam aspek penyampaian informasi dan akurasi model. 
Alternatif solusi yang dipertimbangkan mencakup tiga pendekatan dari literatur dan satu pendekatan yang diusulkan dalam tugas akhir ini.
\begin{enumerate}
    \item \textbf{Alternatif Solusi 1 (A1): Model Kerentanan Karhutla Berbasis Random Forest dan XGBoost} \autocite{Karurung2025}. \\
    \textcite{Karurung2025} mengembangkan model kerentanan kebakaran hutan di provinsi Sumatra, Kalimantan, dan Papua menggunakan algoritma
    RF dan XGBoost dengan berbagai faktor kondisi (meteorologis, biofisik, dan antropogenik). Penelitian ini menghasilkan peta kerentanan
    dan \textit{grid} risiko serta menunjukkan bahwa pemisahan data berdasarkan musim dapat meningkatkan performa model.
    Pendekatan ini efektif dalam pengintegrasian data dan analisis pola kebakaran karena mendapatkan akurasi model yang tinggi. Namun, penyajian yang
    diberikan belum disajikan dalam bentuk visual sederhana sehingga sulit dipahami bagi pengguna.
    \item \textbf{Alternatif Solusi 2 (A2): Sistem Pemantauan dan Informasi \textit{Hotspot} Karhutla SiPongi+} \autocite{sipongi_klhk}. \\
    KLHK melalui SiPongi+ merepresentasikan sistem pemantauan karhutla secara \textit{near real-time} dilengkapi riwayat kejadian dan level risiko kebakaran.
    Sistem ini disajikan melalui perhitungan indikator \textit{hotspot} lewat FWI serta antarmuka peta dan grafik di web serta memiliki respon cepat untuk langkah karhutla secara reaktif. Walaupun begitu, sistem ini 
    masih lebih bersifat \textit{early warning} sehingga belum mampu melakukan analisis pola historis dan integrasi faktor untuk melakukan langkah prediktif
    menggunakan \textit{machine learning} sebagai acuan langkah mitigasi untuk pemangku kepentingan.
    \item \textbf{Alternatif Solusi 3 (A3): Model Kerentanan Karhutla Berbasis Explainable AI (XAI)} \autocite{Abdollahi2023}. \\
    \textcite{Abdollahi2023} mengembangkan model kerentanan kebakaran dengan memanfaatkan teknik XAI untuk menjelaskan kontribusi masing-masing faktor risiko 
    terhadap kerentanan kebakaran. Studi ini menghasilkan grafik kontribusi dari masing-masing hasil yang diprediksi model \textit{machine learning} melalui 
    penjelasan kontribusi faktor. Alternatif ini relevan dengan kebutuhan sistem karena menunjukkan integrasi data karhutla disertai penjelasan pola yang dapat 
    memberikan penjelasan terhadap hasil model yang dapat membantu pengambilan keputusan pemangku kepentingan. Namun penyajian yang dilakukan masih bersifat statis 
    sehingga belum dapat memberikan informasi visual yang mudah digunakan oleh pengguna.
    \item \textbf{Alternatif Solusi 4 (A4): Sistem Informasi Prediksi Karhutla Berbasis \textit{Machine Learning}, XAI, dan Penyajian \textit{Dashboard} Interaktif.} \\
    Alternatif ini merupakan rancangan yang diusulkan dalam tugas akhir ini. Sistem ini dirancang sebagai prototipe sistem analitik berbasis data historis. Sistem ini
    mengintegrasikan data historis \textit{hotspot} dengan faktor lingkungan yang relevan untuk dijadikan analisis prediktif. Hasil prediksi yang diberikan akan 
    dijelaskan melalui XAI untuk menghilangkan aspek "\textit{black box}" dan disajikan dalam bentuk \textit{dashboard} interaktif sehingga dapat membantu pengambilan keputusan 
    mitigasi pengelolaan karhutla oleh pemangku kepentingan.
\end{enumerate}

\subsection{Analisis Penentuan Solusi}
Setelah empat alternatif solusi diidentifikasi, langkah berikutnya adalah menentukan solusi mana yang paling sesuai untuk dikembangkan pada tugas akhir ini.
Penentuan solusi dilakukan berdasarkan pendekatan MCDA agar menghindari aspek subjektif. Secara umum, MCDA digunakan ketika terdapat beberapa alternatif dan kriteria
yang harus dipertimbangkan. Menurut \textcite{OShea2026}, konsep penentuan solkusi dilakukan dengan mendefinisikan kriteria, menentukan bobot pada setiap kriteria, memberikan penilaian terhadap setiap
alternatif, dan menggabungkan penilaian sehingga menjadi satu skor total untuk mengambil keputusan solusi.
Dalam penelitian ini, metode MCDA yang digunakan adalah \textit{Weighted Sum Model} (WSM) dengan alasan kesederhanaan konsep dan transparansi serta cocok untuk jumlah kriteria 
yang tidak terlalu banyak. Konsep WSM dijelaskan pada 
\begin{align}
    S_i = \sum_{j=1}^{n} w_j x_{ij}
    \label{eq:WSM-formula}
\end{align}
dengan :
\begin{itemize}
  \item $S_i$ = skor total alternatif ke-$i$,
  \item $w_j$ = bobot kriteria ke-$j$ (dengan $\sum_{j=1}^{m} w_j = 1$),
  \item $x_{ij}$ = skor alternatif ke-$i$ pada kriteria ke-$j$.
\end{itemize}

Kriteria diambil dari kebutuhan fungsional dan non-fungsional dan disederhanakan menjadi empat kriteria:
\begin{enumerate}
    \item Kesesuaian dengan kebutuhan fungsional (C1): Sejauh mana alternatif mampu memenuhi empat kebutuhan fungsional utama.
    \item Akurasi dan kualitas informasi risiko (C2): Seberapa besar potensi pendekatan tersebut untuk menghasilkan model risiko yang memadai sehingga tidak menyesatkan pemangku.
    Kriteria ini terkait langsung dengan kebutuhan non-fungsional NF1.
    \item Kinerja dan kelayakan implementasi (C3): Seberapa realistis suatu pendekatan diwujudkan sebagai prototipe
    tugas akhir, dilihat dari ketersediaan data, kompleksitas model, dan
    kebutuhan komputasi berdasarkan NF2.
    \item Kemudahan penggunaan (C4): Sejauh mana pendekatan tersebut memberikan penyajian yang mudah dipahami oleh pemangku kepentingan dan membantu mengambil keputusan yang sesuai dengan NF3.
\end{enumerate}

Bobot masing-masing kriteria ditentukan dengan mempertimbangkan tujuan tugas akhir dengan berdasarkan pertimbangan kebutuhan sistem. Berdasarkan pertimbangan tersebut,
bobot yang digunakan adalah sebagai berikut.
\begin{enumerate}
    \item C1 (kesesuaian dengan kebutuhan fungsional): 0,35
    \item C2 (akurasi dan kualitas informasi risiko): 0,25
    \item C3 (kinerja dan kelayakan implementasi): 0,20
    \item C4 (kemudahan penggunaan): 0,20
\end{enumerate}
Bobot tersebut memiliki total 1,00 dan menyesuaikan dengan kebutuhan fungsional dan non-fungsional. Dari keempat alternatif yang diajukan, penilaian dilakukan
dengan menggunakan nilai kriteria yang telah ditentukan sebelumnya. Perhitungan yang dihasilkan akan menentukan hasil akhir yang bertujuan sebagai penentuan solusi.
Penilaian masing-masing alternatif dilakukan dengan melihat kriteria pada skala 1-5 dengan penjelasan angka 1 berarti sangat kurang dalam memenuhi kriteria dan angka 5
berarti sangat memenuhi kriteria. Skor ditentukan secara kualitatif berdasarkan kelebihan dan kekurangan masing-masing alternatif yang telah dijelaskan sebelumnya pada Subbab III.2.1 Alternatif Solusi.

Secara ringkas, A1 dinilai kuat pada pemodelan risiko namun belum menyediakan sistem informasi yang siap pakai. A2 dinilai kuat pada penyajian pemantauan operasional, namun 
sangat lemah pada analisis historis dan pemodelan prediktif. A3 cenderung kuat pada akurasi dan memiliki interpretabilitas pada hasil prediksi, tetapi visualisasi yang diberikan 
masih statis dan fokusnya cenderung pada hasil analisis. A4 dirancang khusus untuk menyeimbangkan kekurangan dari keempat kriteria dalam konteks provinsi Kalimantan. Matriks penilaian
alternatif dan skor total dihitung menggunakan rumus WSM sesuai pada persamaan III.1 dan dirangkum pada Tabel III.3
\begin{table}[h]
  \centering
  \begin{tabular}{|p{1.8cm}|c|c|c|c|c|}
    \hline
    Alternatif & C1 & C2 & C3 & C4 & Skor total $S_i$ \\ \hline
    Bobot $w_j$ & 0{,}35 & 0{,}25 & 0{,}20 & 0{,}20 & -- \\ \hline
    A1 & 4 & 4 & 3 & 3 & 3{,}60 \\ \hline
    A2 & 3 & 2 & 2 & 4 & 2{,}75 \\ \hline
    A3 & 4 & 4 & 3 & 4 & 3{,}80 \\ \hline
    A4 & 5 & 4 & 4 & 4 & 4{,}35 \\ \hline
  \end{tabular}
  \caption{Matriks penilaian alternatif solusi menggunakan WSM}
  \label{tbl:mcda-matrix}
\end{table}

Berdasarkan Tabel~\ref{tbl:mcda-matrix}, Alternatif Solusi 4 (A4)
memperoleh skor total tertinggi ($S_{A4} = 4{,}35$) sehingga dipilih
sebagai solusi yang akan dikembangkan pada tugas akhir ini. A4 dinilai
paling seimbang dalam memenuhi kebutuhan fungsional dan non-fungsional,
serta masih realistis untuk diwujudkan sebagai prototipe sistem analitik
berbasis data historis untuk Kalimantan.

Alternatif lain tetap dimanfaatkan sebagai rujukan: A1 dan A3 sebagai
acuan pemodelan dan pendekatan XAI, sedangkan A2 sebagai gambaran
kondisi sistem pemantauan karhutla yang sudah ada (SiPongi+) dan dasar
untuk mengidentifikasi kebutuhan analisis yang belum terpenuhi.
\section{\textit{Data Understanding}}
Dalam mengembangkan solusi, langkah awal yang dilakukan adalah mencari dan memahami data-data yang tersedia 
untuk memahami konteks permasalahan dengan lebih dalam. Selain itu, data juga bermanfaat untuk melatih model 
prediktif sehingga dapat dimanfaatkan dalam sistem. Pencarian data bersifat wajib untuk langkah awal, namun 
terkadang data yang digunakan belum cukup sehingga pencarian data selanjutnya bersifat iteratif. Pada subbab ini,
data yang akan dipahami adalah data utama yang akan digunakan pada tugas akhir ini, yaitu data \textit{hotspot} 
karhutla dari \textit{National Aeronautics and Space Administration} (NASA) bernama \textit{Fire Information for Resource
Management System} (FIRMS). Data tersebut merupakan data yang akan menjadi label atau target karena berisikan titik letak terjadinya \textit{hotspot}.
\subsection{Deskripsi Umum \textit{Dataset}}
Data FIRMS diperoleh dalam format \textit{shapefile} (shp) yang dikonversi ke file \textit{Coma Separated Values} (csv) dengan
memanfaatkan \textit{Quantum Geographic Information System} (QGIS). QGIS merupakan perangkat lunak GIS yang gratis, \textit{open-source}, 
dan multifungsi yang memungkinkan pengguna untuk membuat, mengedit, memvisualisasikan, menganalisis, dan mempublikasikan data spasial atau
geografis. Data FIRMS yang diperoleh berupa produk \textit{Visible Infrared Imaging Radiometer Suite} (VIIRS) 375 m \textit{Active Fire} untuk 
satelit \textit{National Oceanic and Atmospheric Administration} (NOAA-20). Satelit ini memiliki keunggulan berupa resolusi spasial 
yang lebih tinggi dibandingkan \textit{Moderate Resolution Imaging Spectroradiometer} (MODIS) 1 km sehingga mampu mendeteksi panas 
berskala kecil seperti kebakaran lahan gambut atau pembukaan lahan dengan api.

\textit{Dataset} ini diambil pada situs resmi NASA FIRMS. \textit{Dataset} ini berisi 846.831 baris data \textit{hotspot}
dengan 15 fitur utama yang mencakup periode observasi dari 1 April 2018 sampai 1 Januari 2023. Pemilihan lokasi data befokus pada
wilayah Indonesia. Untuk memberikan gambaran awal tentang data, Tabel III.4 menunjukkan deskripsi singkat tentang seluruh fitur beserta
contoh nilai dari masing-masing fitur.
\begin{table}[h]
\centering
\label{tbl:deskripsi-VIIRS}
\caption{Deskripsi atribut pada \textit{dataset} VIIRS 375 m}
\begin{tabular}{|p{3cm}|p{1cm}|p{6cm}|p{2cm}|}
\hline
\textbf{Nama Fitur} & \textbf{Tipe Data} & \textbf{Deskripsi Singkat} & \textbf{Contoh Nilai} \\
\hline
LATITUDE    & \textit{float} & Lintang geografis titik \textit{hotspot} (derajat, WGS84) & --3.85 \\
LONGITUDE   & \textit{float} & Bujur geografis titik \textit{hotspot} (derajat, WGS84)  & 136.36 \\
BRIGHTNESS  & \textit{float} & \textit{Brightness temperature} kanal termal utama VIIRS (Kelvin) & 330.34 \\
SCAN        & \textit{float} & Ukuran piksel pada arah \textit{scan} satelit (derajat) & 0.40 \\
TRACK       & \textit{float} & Ukuran piksel pada arah \textit{track} satelit (derajat) & 0.37 \\
ACQ\_DATE   & \textit{string} & Tanggal akuisisi dalam UTC (format YYYY/MM/DD) & 2018/04/01 \\
ACQ\_TIME   & \textit{int}    & Waktu akuisisi dalam UTC (HHMM) & 416 \\
SATELLITE   & \textit{string} & Kode satelit penginderaan; seluruh record berisi N20 (NOAA-20) & N20 \\
INSTRUMENT  & \textit{string} & Nama sensor; seluruh record berisi VIIRS & VIIRS \\
CONFIDENCE  & \textit{string} & Tingkat kepercayaan deteksi \textit{hotspot} (l=low, n=nominal, h=high) & n \\
VERSION     & \textit{int}    & Versi produk FIRMS; seluruh record berisi 2 (Collection 2) & 2 \\
BRIGHT\_T31 & \textit{float} & \textit{Brightness temperature} kanal referensi T31 (Kelvin) & 297.54 \\
FRP         & \textit{float} & \textit{Fire Radiative Power} (MW), indikator intensitas kebakaran & 3.93 \\
DAYNIGHT    & \textit{string} & Penanda kondisi siang/malam saat deteksi (D = day, N = night) & D \\
TYPE        & \textit{int}    & Kode jenis deteksi sesuai definisi FIRMS & 0 \\
\hline
\end{tabular}
\end{table}

\subsection{Kelengkapan dan Distribusi Data}
Pemeriksaan awal terhadap struktur data adalah dengan menunjukkan bahwa seluruh fitur tidak memiliki nilai 
kosong (\textit{missing values}) dan data duplikat. Hasil analisis menemukan bahwa data tidak memiliki nilai kosong 
dan tidak ditemukan baris duplikat. Selanjutnya, analisis distribusi data dilakukan dengan melihat jumlah \textit{hotspot}
berdasarkan data tahunan. Distribusi jumlah \textit{hotspot} per tahun ditunjukkan pada 2.
\begin{figure}[h] % pilihan opsi yang disarankan: t = top, b = bottom, h = here
	\centering
  \captionsetup{justification=centering}
    	\includegraphics[width=1\textwidth]{image/data-persebaran-VIIRS.png}
	\caption{Jumlah \textit{hotspot} per tahun (VIIRS NOAA-20)}
	\label{gambar:persebaran-VIIRS}
\end{figure}

Pada Gambar \ref{gambar:model-konseptual-saat-ini}, jumlah data \textit{hotspot} yang terjadi pada tahun 2023 sangat kecil. Hal ini tidak sejalan dengan pola tahunan
sebelumnya maupun statistik kebakaran dari sumber resmi yang ditujunkkan pada Bab 1. Hal ini bisa jadi mengindikasikan 
bahwa data tahun 2023 pada \textit{dataset} ini kemungkinan tidak lengkap disebabkan hanya mengambil bulan Januari pada tanggal 1. Oleh
karena itu, pada tahap pemodelan selanjutnya, data tahun 2023 direncanakan untuk tidak digunakan. Namun, data ini akan tetap dilaporkan 
sebagai bagian dari kondisi \textit{dataset}. 

Analisis distribusi selanjutnya yang dilakukan adalah dengan menilai distribusi dari fitur tingkat kepercayaan deteksi (\textit{confidence}).
Hal ini bertujuan untuk memahami bagaimana \textit{dataset} ini merepresentasikan titik api karena seluruh titik api yang muncul dan terdeteksi
oleh satelit akan dimasukkan sebagai data. Ini bisa menjadi kesalahpahaman karena bisa saja titik api yang ditunjukkan bukanlah sebuah kebakaran, 
namun hanya percikan api yang mungkin disebabkan oleh aktivitas manusia. Distribusi intensitas api ditunjukkan pada Gambar \ref{gambar:persebaran-VIIRS}.
\begin{figure}[h] % pilihan opsi yang disarankan: t = top, b = bottom, h = here
	\centering
  \captionsetup{justification=centering}
    	\includegraphics[width=1\textwidth]{image/data-persebaran-confidence.png}
	\caption{Data persebaran fitur \textit{confidence}}
	\label{gambar:persebaran-confidence}
\end{figure}

Pada Gambar \ref{gambar:persebaran-confidence}, dominasi kategori nominal menunjukkan bahwa sebagian besar deteksi berada pada tingkat kepercayaan standar yang disarankan oleh FIRMS.
Data dengan label \textit{low} menunjukkan bahwa titik tersebut bisa ditandai sebagai kandidat \textit{noise} disebabkan titik api yang dimunculkan bukan merupakan kebakaran. Maka dari itu, pada pengembangan selanjutnya persiapkan data akan memanfaatkan batas kepercayaan yang sesuai agar dapat
memberikan informasi kebakaran yang relevan.


% \section{Analisis Kondisi Saat Ini}
% Menurut \textcite{laudon2020}, gambarkan terlebih dahulu model konseptual sistem yang ada saat ini. Model konseptual ini berisi berbagai komponen atau subsitem dan interaksi antarsubsistem tersebut. Setelah itu, berikan penjelasan tentang masalah yang ada pada sistem tersebut. Paragraf berikut berisi contoh penjabaran masalah sistem informasi fasilitas kesehatan untuk pasien \autocite{pressman2019}. 
% \section{Analisis Kebutuhan}
% \lipsum[4]
% \subsection{Identifikasi Masalah Pengguna}
% \lipsum[5]
% \subsection{Kebutuhan Fungsional}
% \lipsum[6]
% \subsection{Kebutuhan Nonfungsional}
% \lipsum[7]

% \section{Analisis Pemilihan Solusi}
% \subsection{Alternatif Solusi}
% \lipsum[8]
% \subsection{Analisis Penentuan Solusi}
% \lipsum[9]